\documentclass{article}
\usepackage[utf8]{inputenc}
\usepackage[euler]{textgreek}
\usepackage{amsmath}
\usepackage{amssymb}
\newcommand\tab[1][4mm]{\hspace*{#1}}

\title{ADA Assignment 5}
\author{Ankur Sharma}
\date{April 2018}

\begin{document}

\maketitle
\section*{}
Partners: Anvit Mangal (2016135) and Ishaan Bassi (2016238).

\section*{Problem 1}
Given a set $\{x_1, x_2, . . . , x_n\}$ of points on the real line, determine the smallest set of unit-length closed intervals (e.g. the interval [x, x+1]) that contains all of the points. Give the most efficient algorithm you can to solve this problem, prove it is correct and analyze the time complexity. \\
\vspace{1mm} \\
\textbf{Solution:} \\  
Let X be the given set of real numbers $\{x_1, x_2, . . . , x_n\}$ with size equal to n and R be the required smallest set of unit-length intervals s.t. it contains all of point in X. \\
From inspection, we can see that \textbf{A Greedy Approach} can be applied here in order to get the desired result. A rough idea is to find the smallest element($x$) in X, and put the interval $[x, x+1]$ in R and removing all the numbers that lie inside $[x, x+1]$ from X. We do this repeatedly till X gets empty. \\
Now, let's to compute the time-complexity of above discussed approach. While X is not empty: First we find the smallest element(x) in X which would take $O(n)$ time (Since X is not sorted) and then, finally removing points from X that lie inside $[x, x+1]$ which would again take $O(n)$ time. Hence making it $O(n^2)$ in the worst case when 'while loop' runs n times. \\
Can we think of a better way? Yes! If X was a sorted set(arraylist), then over-all time-complexity can be reduced. Therefore, we first sort X and then do the same computations. Following is the Pseudo-Code for the same. \\
\vspace{0.5mm} \\
\textbf{Procedure} getSetOfIntervals(X): \\
\tab $\#\ sort\ X\ using\ mergesort\ algorithm$ \\
\tab S = mergeSort(X) $\#\ S = \{s_1, s_2, . . . , s_n\}$ \\
\tab $\#\ S\ is\ a\ sorted\ set(arraylist)\ s.t.\ S[0]\ is\ the\ smallest\ element\ in\ S$ \\
\tab set R = $\phi$ \\
\tab while S is not empty: \\
\tab\tab s = S[0] \\
\tab\tab I = [s, s+1] \\
\tab\tab R.add(I) \\
\tab\tab remove points from S that lie inside I starting with s \\
\tab return R \\
\vspace{0.5mm} \\
\textbf{Proof of Correctness:} \\
By way of contradiction, suppose that R is not a correct set returned by above mentioned procedure. Let T be the correct required set. Let $q = [t, t+1]\ \epsilon$ T s.t. $s_1\ \epsilon\ q$. Now, since $s_1$ is the smallest element in X $\rightarrow$ there are no points on the left of $s_1$ in X $\rightarrow\ t = s_1$ and therefore, $q = [s_1, s_1+1]$. Moving on \textbf{inductively}, we get that R is same as T b/c once we remove points from S that lie in $[s_1, s_1+1]$ and we proceed inductively by same argument. Hence, R = T. \\
Algorithm terminates: We start with merge-sort which is a trivial instruction and then we come to 'while loop'. Since we are removing atleast one element(more precisely, atleast smallest one) from S in 'while loop'. Therefore, 'while loop' runs atmost n times. 
\vspace{0.5mm} \\
\textbf{Time-Complexity:} \\
First we have merge-sort which takes $O(n*log\ n)$. 
'While loop' runs for atmost n times which is the worst-case when we only remove one element from S at each iteration $\rightarrow$   there are n intervals in R. Hence, $O(n)$ here. And therefore, $O(n*log\ n + n)$ gives us $O(n*log\ n)$. 

\section*{Problem 2}
Consider the problem of making change from n cents using the fewest coins when the available coins are quarters, dimes, nickels and pennies. Design a greedy algorithm for this problem and prove its correctness. Also analyze the running time of your algorithm. \\
\vspace{1mm} \\
\textbf{Solution:} \\
Available coins: quarters(=25 cents), dimes(=10 cents), nickels(=5 cents) and pennies(= 1 cent). Let $n$ be the given no. of cents. Clearly, if $n < 5$, we are forced to use pennies (b/c for a nickel we need 5 cents) and if $5 <= n < 10$, we will go for nickels etc. So, the idea is to use quarter whenever we can else dimes $>$ nickels $>$ pennies acc. to n. Pseudo-code is as follows: \\
\textbf{Procedure} minimumCoin($n$):\\
\tab coins = 0 $\#\ total\ no.\ of\ coins$ \\
\tab if $n < 5$ : \\
\tab\tab  coins = n $\#\ n\ pennies $\\ 
\tab else if $n < 10$ : \\
\tab\tab coins = 1 + n - 5 $\#\ 1\ nickel\ and\ n - 5\ pennies $\\ 
\tab else if $n < 25$ : \\
\tab\tab coins = n25(n) \\
\tab else: \\
\tab\tab q = Math.floor(n/25) \\
\tab\tab coins = q  $\#\ q\ quarters\ and\ ..$\\ 
\tab\tab n -= q*25 \\
\tab\tab if $n < 5$ : \\
\tab\tab\tab  coins += n $\#\ n\ pennies $\\ 
\tab\tab else if $n < 10$ : \\
\tab\tab\tab coins += 1 + n - 5 $\#\ 1\ nickel\ and\ n - 5\ pennies $\\ 
\tab\tab else if $n < 25$ : \\
\tab\tab\tab coins += n25(n) \\
\vspace{0.5mm} \\
\textbf{Procedure} n25(n): \\
\tab d = Math.floor(n/10) \\
\tab\tab coins = d $\#\ d\ dimes\ and ..$\\
\tab\tab n -= d*10 \\
\tab\tab if $n < 5$ : \\
\tab\tab\tab  coins += n $\#\ n\ pennies $\\ 
\tab\tab else if $n < 10$ : \\
\tab\tab\tab coins += 1 + n - 5 $\#\ 1\ nickel\ and\ n - 5\ pennies $ \\
\tab return coins \\
\vspace{0.5mm} \\
\textbf{Proof of Correctness:}\\
The idea is to use coins s.t. it has the maximum value(cents). Therefore, we use quarters first if possible, then dimes if possible, then nickel if possible, then pennies. \\
Claim: Our algorithm works in the same manner as the idea suggests. \\
If no. of pennies are greater than or equal to 25, function minimumCost(n) enters in the conditional statement where $n>=25$ i.e. else. Similarly, follows for dimes from the code. Q.E.D. \\
\vspace{1mm} \\
\textbf{Time-Complexity:} $O(1)\;\; (\because\ there\ are\ only\ if-else\ statements.) $

\section*{Problem 3}
The HAM-PATH problem is the following: Given an undirected graph G, is there a path in G that visits all vertices exactly once. The HAM-CYCLE problem is the following: Given an undirected graph G, is there a cycle in G that visits all vertices exactly once. Give a sketch of a proof that HAM-PATH is in NP. Now show that HAM-PATH is NP-complete by reducing HAM-CYCLE to HAM-PATH. (Assume the HAM-CYCLE is NP-HARD.) \\
\vspace{1mm} \\
\textbf{Solution:} \\
\textbf{a.} To prove: HAM-PATH belongs to NP. \\
Proof: Let G be a graph. It is enough to prove that if there exists a path s.t. it visits all vertices of G exactly one. Let no. of vertices in G = n. Take any ordering of vertices (an ordered set) C. Now, verify if given ordering is correct i.e. $\exists$ an edge between consecutive vertices in given ordering(or Certificate). Verification of any given ordering can be done in polynomial time. Note: All possible orderings are tried non-deterministically. Hence, overall problem reduces to polynomial time. \\
Algorithm: \\
Choose an ordering C=$\{v_1, v_2, ... ,v_n\}$ non-deterministically. \\
for i in range (1, n-1): \\
\tab if $\exists$ an edge between $v_i$ and $v_{i+1}$: \\
\tab\tab $\#\ do\ nothing.$ \\
\tab else: \\
\tab\tab return false \\
return true \\
Hence, it's done in $O(n)$ time i.e. polynomial therefore, HAM-PATH belongs to NP. \\
\vspace{1mm} \\
\textbf{b.} To prove: HAM-PATH is NP-complete. \\
Given: HAM-CYCLE is NP-HARD. \\
Proof: We know that if some language L is in NP and NP-HARD, then L is NP-COMPLETE. $\because$ in part \textbf{a.}, we proved that HAM-PATH is in NP. $\therefore$ it is enough to prove that HAM-PATH is NP-HARD. If we can reduce HAM-CYCLE to HAM-PATH, this implies HAM-PATH is NP-HARD because HAM-CYCLE is NP-HARD (given). \\
Let G be a graph. We will solve HAM-CYCLE problem using HAM-PATH. Now, construct a graph G' as follows: \\
1) G' = G \\
2) Pick any arbitrary vertex $v$ from G and add another vertex $v'$ (a copy of $v$ with all edges of $v$) to G'. \\\
3) Add two more vertices $x$ and $y$ to G' s.t. $\exists$ an edge between $v$ and $x$, and $\exists$ an edge between $v'$ and $y\ \rightarrow$ degree of $x$ = degree of $y$ = 1. \\
We claim that: HAM-PATH in G' $\rightarrow$ HAM-CYCLE in G. \\
As $\exists$ HAM-PATH in G', it must visit $x$ and $y$ but deg($x$) = deg($y$) = 1 $\rightarrow$ HAM-PATH begins $x$ and ends with $y$. As $x$ is connected to $v$ and $y$ is connected to $v'$, removing $x$ and $y$ from G' would change HAM-PATH. Therefore, now HAM-PATH in new G' begins with $v$ and ends with $v'$. Let $e$ be an edge in HAM-PATH s.t. $e$ connects $v'$ to some vertex $t$. Now, remove $v'$ $\rightarrow$ HAM-PATH begins with $v$ and ends with $t$. But $v'$ is a copy of $v$, therefore $\exists$ an edge from $t$ to $v$ hence, making a cycle in G. Q.E.D.

\section*{Problem 4}
Given an integer k, divide a set of n objects into k coherent clusters such that spacing, i.e., Min distance between any pair of points in different clusters, is maximized. Write an efficient algorithm for this problem. \\
\vspace{1mm} \\
\textbf{Solution:} \\
K-Means Clustering can be used to achieve the required result. \\
Let given set of points be X=$\{x_1, x_2, . . . , x_n\}$. Since we want to divide this set into k coherent clusters, each cluster will have a centroid $c_i$ s.t. $c_i$ belongs to $i^{th}$ cluster (where $1<=i<=k$). First, we will randomly place our $c_i$s(or initialize them with random co-ordinates) initially. After that, for every $x_j$ we assign a centroid which is nearest(Euclidian distance) to that (where $1<=j<=n$). Then, we optimize the co-ordinates of each centroid by taking mean of euclidian distance from $c_i$ to $x_j$s (to which $c_i$ belongs) for all dimensions individually. We repeat this until no $x_j$ gets assigned to a new $c_i$. \\
\textbf{Procedure} K-Means Clustering Algorithm(X, k): \\
\tab C : 2-D arraylist s.t. arraylist c[i] stores a list of $x_j$s assigned to $i^{th}$ centroid. \\
\tab A : 1-D array s.t. X[j] = centroid assigned to $j^{th}$ point in X. \\
\tab for i in range(1, k): \\
\tab\tab $c_i$=RandomCoordinate() \\
\tab while (no convergence) : \\
\tab\tab for j in range(1,n): \\
\tab\tab\tab A[j]=getNearestCentroid($x_j$) $\#\ acc.\ to\ euclidian\ dist.$ \\
\tab\tab\tab C[A[j]].add($x_j$) \\
\tab\tab for i in range(1, k): \\
\tab\tab\tab for each point x in C[i]: \\
\tab\tab\tab\tab mean+= getEuclidianDistance($c_i$, $x$) $\#\ for\ each\ dimension\ seperately.$ \\
\tab\tab\tab mean/=size(C[i]) \\
\tab\tab\tab $c_i$ = mean (d-Dimension vector) \\


\section*{Problem 5}
Prove the cut property and the cycle property for the MST.
\vspace{1mm} \\
\textbf{Solution:} \\
\textbf{To Prove: Cut Property for the MST-} For any cut C of the graph, if the weight of an edge e in the cut-set of C is strictly smaller than the weights of all other edges of the cut-set of C, then this edge belongs to all MSTs of the graph. (as stated on wikipedia) \\
\textbf{Proof:} \\
Let C be a cut of a graph and edge $e$ belongs to the cut-set C s.t. weight of $e$ is strictly smaller than the other edges in cut-set C. \\
By way of Contradiction, let $e$ does not belong to some MST M. Now, add $e$ to M thereby, creating a cycle in M (Since M is a tree). Now, we remove an edge $e'$ from generated cycle s.t. it also belongs to the cut $e$ belongs to. Since, $e$ and $e'$ belong to the same cut s.t. $e$ is an edge with strictly smaller weight than the other. Hence, after removing $e'$, we get an MST with smaller weight than M which is a contradiction. Q.E.D. \\
\vspace{0.5mm} \\
\textbf{To Prove: Cycle Property for the MST-} For any cycle C in the graph, if the weight of an edge e of C is larger than the individual weights of all other edges of C, then this edge cannot belong to a MST. (as stated on wikipedia) \\
\textbf{Proof:} \\
Let C be a cycle in a graph and edge $e$ belongs to the cycle C s.t. weight of $e$ is larger than the other edges in cycle C. \\
By way of Contradiction, let $e$ belongs to some MST M. Removing $e$ from M would disconnect the M into two sub-trees. Since, $e$ belongs to cycle C, there exists an edge $e'$ in C s.t. adding that edge to M would again connect the sub-tress and thereby making a new MST. But weight of $e$ is larger than $e'$ implying new MST has weight less than M which is a contradiction. Q.E.D.

\section*{Problem 6}
You have 2 random variables $X_1$ and $X_2$. You compute 500 $Y$ values following the equation: $Y = aX_1 + bX_2 + c+$ small Gaussian noise where $a = b = c = 2$. Now consider that you do not know values of $a$, $b$ and $c$. Consider $a$, $b$ and $c$ as unknowns. Given the 500 $Y$ values, try to estimate those using gradient descent such that the plane fits the points best. \\
\vspace{1mm} \\
\textbf{Solution:} \\
It is a simple problem that can be solved using gradient descent algorithm which essentially minimize the cost function determined by \\
	\[ J(a, b, c) = \frac{1}{2*500} \sum_{i=1}^{500} (f_i(a, b, c) - Y_i)^2 = \frac{1}{1000} \sum_{i=1}^{500} (f_i(a, b, c) - Y_i)^2 \] \\
	 \[Let\ J_a = \frac{\partial J}{\partial a} = \frac{1}{500} \sum_{i=1}^{500} (f_i(a, b, c) - Y_i)*X_{1i}, \] \\
\[J_b = \frac{\partial J}{\partial b} = \frac{1}{500} \sum_{i=1}^{500} (f_i(a, b, c) - Y_i)*X_{2i}, \] \\
\[J_c = \frac{\partial J}{\partial c} = \frac{1}{500} \sum_{i=1}^{500} (f_i(a, b, c) - Y_i) \] \\
\textbf{Procedure} gradientDescent: \\
a = RandomNumber(), b = RandomNumber(), c = RandomNumber(); \\
$\alpha$ = 0.1 $\#\ learning\ rate$ \\
%J(a, b, c) = $\frac{1}{2*500} \sum_{i=1}^{500} (f_i(a, b, c) - Y_i)^2$ = $\frac{1}{1000} \sum_{i=1}^{500} (f_i(a, b, c) - Y_i)^2$ \\
while (no convergence): \\
\tab a = a - $(\alpha * J_a)$ \\
\tab b = b - $(\alpha * J_b)$ \\
\tab c = c - $(\alpha * J_c)$ \\
return a, b, c;


\end{document}